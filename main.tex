\documentclass[12pt,a4paper]{article}

\usepackage{fullpage}
\usepackage{indentfirst}
\usepackage{parskip}
\usepackage{amsmath}
\usepackage{hyperref}
\usepackage{bm}
\usepackage{enumerate}
\usepackage{enumitem}
\usepackage{graphicx}
\usepackage{booktabs}
\usepackage[UTF8]{ctex}

\begin{document}
\setlength{\parindent}{2em}

\pdfbookmark[0]{Front page}{label:frontpage}%
\begin{titlepage}
  \noindent%
  \begin{tabular}{@{}p{\textwidth}@{}}
    \toprule[2pt]
    \midrule
    \vspace{0.2cm}\\
    \begin{center}
    \Huge{\textbf{
      基于AODV的支付通道网络\\路由协议 \\[8pt]
    }}
    \end{center}
    \begin{center}
      \Large{
        分布式计算课程主题报告
      }
    \end{center}
    \vspace{0.2cm}\\
    \midrule
    \toprule[2pt]
  \end{tabular}
  \vspace{5 cm}
  \begin{center}
    {\Large
      第15组
    }\\
    \vspace{0.2cm}
    {\Large
      1831605 \quad 刘诗洋 \\[5pt]
      1831606 \quad 陆思远 \\[5pt]
      1831607 \quad 余\quad豪 \\[5pt]
    }
  \end{center}
  \vfill
  \begin{center}
    {\Large
      同济大学\\[5pt]
      软件学院\\[5pt]
    }
  \end{center}
\end{titlepage}
\clearpage

\tableofcontents
\clearpage

\section{概述}
1. 比特币及区块链衍生品出现及其目的;

2. 区块链存在一些共有的问题;

3. 为了解决这些问题,提出off-chian solutions. Lightning Network是其中代表;

4. PCN 用于连接链下的用户,用户间形成的网络。网络面临问题: 寻找最优路径(基于路由费用、费率、可靠性)
LN使用了主动式的路由协议。有一些缺点。

5. 我们提出的协议,可以解决这些缺点。具体为...

\section{领域研究}
1. 路由相关定义

2. PSTN,MANET,
由于PCN研究较少,故参考MANET(有一些相似之处)

3. 路由协议主要有5种,reactive, proactive, hybrid (i.e. a combination out of both: reactive and proactive), hierarchical and coordinate-based。

4. Flare,hybrid network. 列举我们的目标与Flare的区别。

\section{需求与算法选择}
为了找到合适的算法,对网络提出了几点需求:
1. Autonomy and self-reliance
2. Cost guaranties: Each
3. Time-lock guaranties
4. Flexibility
5. Prevent network partitioning
6. Real-time
7. Up-to-dateness
8. Lightweight and scalable
9. Trustlessness
10. Trustlessness
基于以上,我们选择/设计了AODV-based algorithm.
\end{document}